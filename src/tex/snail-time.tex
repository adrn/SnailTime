% Notes:
% -

% Relevant papers (for bibliograph):
% https://arxiv.org/pdf/1903.01493.pdf
% https://arxiv.org/pdf/2009.02434.pdf
% https://arxiv.org/pdf/1809.02658.pdf
% https://arxiv.org/pdf/1902.10113.pdf
% https://arxiv.org/pdf/2011.02490.pdf
% https://arxiv.org/pdf/1808.00451.pdf

% \begin{figure}[!t]
% \begin{center}
% % \includegraphics[width=0.9\textwidth]{visitstats.pdf}
% {\color{red} Figure placeholder}
% \end{center}
% \caption{%
% TODO
% \label{fig:chiplots}
% }
% \end{figure}

\PassOptionsToPackage{usenames,dvipsnames}{xcolor}
\documentclass[modern]{aastex631}
% \documentclass[twocolumn]{aastex631}
\usepackage{showyourwork}

% Load common packages
\usepackage{microtype}  % ALWAYS!
\usepackage{amsmath}
\usepackage{amsfonts}
\usepackage{amssymb}
\usepackage{booktabs}
\usepackage{graphicx}
% \usepackage{color}

\usepackage{enumitem}
\setlist[description]{style=unboxed}

% Some style hacks:
\renewcommand{\twocolumngrid}{\onecolumngrid}
\setlength{\parindent}{1.1\baselineskip}
\addtolength{\topmargin}{-0.2in}
\addtolength{\textheight}{0.4in}
\sloppy\sloppypar\raggedbottom\frenchspacing

\graphicspath{{figures/}}
\input{preamble.tex}

% Custom:
\newcommand{\kepler}{\project{Kepler}}
\newcommand{\ktwo}{\project{K2}}
\newcommand{\tess}{\acronym{TESS}}
\newcommand{\numax}{\ensuremath{\nu_{\rm max}}}

\shorttitle{}
\shortauthors{Price-Whelan et al.}

\begin{document}

\title{Timing Vertical Phase Spirals with Stellar Lifetimes}

\newcommand{\affcca}{
    Center for Computational Astrophysics, Flatiron Institute, \\
    162 Fifth Ave, New York, NY 10010, USA
}

\author[0000-0003-0872-7098]{Adrian~M.~Price-Whelan}
\affiliation{\affcca}
\email{aprice-whelan@flatironinstitute.org}
\correspondingauthor{Adrian M. Price-Whelan}

% TODO: orcid
\author{Jason A. S. Hunt}
\affiliation{\affcca}

% TODO: orcid, aff
% \author{Elise Darragh-Ford}
% \affiliation{\affstanford}
% \affiliation{\affcca}
\
% TODO: orcid, affs
% \author{David~W.~Hogg}
% \affiliation{\affcca}
% \affiliation{\affnyu}
% \affiliation{\affmpia}

% TODO: orcid, aff
% \author{Kathryn Johnston}
% \affiliation{\affcolumbia}

\author{+ more}


\begin{abstract}\noindent
% Context
The discovery of phase-space spiral features in the Milky Way disk --- made possible by exquisite stellar astrometry from the \gaia\ Mission --- has enabled new methods for constraining the mass distribution and perturbation history of the Galaxy.
These spirals likely formed as a result of phase mixing of orbits with different characteristic frequencies as a result of perturbations to the Galactic disk.
As a result, the amplitude, morphology, and global dependence (over the Galactic disk), of vertical ($z$--$v_z$) phase spirals is determined by the time, timescales, and strengths of perturbations to the Milky Way and the resulting response of the system.
Mapping and dynamically rewinding the observed phase spirals has therefore led to new insights about the orbit and interactions of the Sagittarius dwarf galaxy (a significant perturber of the Milky Way) and the subsequent evolution of the Galaxy.
% Aims
In this work, we aim to measure the onset time of vertical phase spirals in dynamically-separated populations of stars that are presently near the Sun by exploiting main-sequence stellar lifetimes rather than dynamical phase mixing.
% Methods
We use stellar temperatures derived from \gaia\ Data Release 3 (\dr{3}) and main sequence lifetimes predicted from the \placeholder{TODO} stellar models to split our stellar samples into subpopulations with consistent maximum stellar lifetimes.
In these subpopulations, we measure the amplitude of non-phase-mixed structure in the vertical kinematics using empirical matched filters tuned to the orbit structure of a given dynamical patch, without assuming a form for the gravitational potential of the Milky Way.
% Results
We find that the ...
% Conclusions
This work presents a new approach for quantifying the onset time of vertical phase-space spirals in the Milky Way leveraging stellar physics rather than dynamics.


\end{abstract}

% \keywords{}

\section*{~}\clearpage

\section{Introduction} \label{sec:intro}

% Milky Way context and mergers and cosomology crap. Want to measure dark matter/mass distribution, and formation history/evolution of galaxy.
% The Milky Way is a unique laboratory for studying the dynamical processes that shape galaxies and the detailed structure of dark matter within a galaxy.
% It is the one galaxy where contemporary stellar surveys measure kinematics and stellar parameters for hundreds of millions to billions of individual stars at all evolutionary stages, and where our perspective within the Galaxy grants us a three-dimensional view of its stars (and gas).
The Milky Way (MW) is a uniquely important galaxy, for one because it is our home, but
also because it serves as a benchmark system for contextualizing studies of external
galaxies.
Significant effort has therefore gone into modeling the observed distribution and
kinematics of the MW's stellar populations \citep[for a review,][]{Bland-Hawthron:2016}
and using these observations to build dynamical models of its structure and evolution
\citep[e.g.,][]{Binney:2008, CITE, MANY}.
However, we observe the MW and its local environment at (effectively) a single dynamical
snapshot.
This has historically meant that many dynamical measurements of the MW's properties have
had to rely on assumptions that the Galaxy is (at least approximately) in steady state,
the stellar distribution function is highly symmetric, or that any perturbations and
time-dependent processes can be (statistically) averaged over.
In recent years, owing to precise and expansive data from stellar surveys like the
\gaia\ Mission \citep{Gaia-overview}, it is now known that many of these assumptions
are significantly violated.

% However, explicitly modeling these departures from steady state provide promising new directions for precisely constraining the dark matter distribution around the Galaxy, measuring the global structural parameters of the stellar components of the Milky Way, and inferring the mass assembly history and dominant dynamical processes.
% A critical step toward making these precise measurements is to characterize the TODO ... of signatures of disequilibrium ... in data.

Recent observations (and simulations) of the Galaxy have demonstrated the existence (or
expectation) of un-mixed kinematic substructure at all spatial scales in the Galaxy.
For example, on the largest spatial scales in the Galaxy, the infall of the Magellanic
Clouds is thought to induce a global dipole mode in the dark matter distribution
\citep[e.g.,][]{Garavito-Camargo, blah}, a wake and over-density in the dark matter and
stellar halo \citep{Garavito-Camargo, Conroy}, and a radius-dependent reflex motion of
the inner MW with respect to the outer galaxy \citep{Garavito-Camargo, Petersen, Erkal}.
On smaller spatial scales, significant substructure is seen in the distribution function
of stars in regions of the Galactic disk \citep[e.g.,][]{hipparcos, dehnen, Trick} and
in the morphologies of stellar streams \citep{??, PWB2018, Li-AAU}.
These smaller-scale signatures of time-dependence are interpreted either as a result of
complex resonant dynamics with massive patterns \citep[e.g., spiral arms or the galactic
bar][]{Hunt-transient-spirals, D'onghia bar, Pearson-bar-streams} or as a result of
transient perturbations from massive bodies \citep[e.g., dark matter subhalos or
satellite galaxies][]{Bonaca:2019, ...}.

In the MW disk --- the focus of this work --- signatures of disequilibrium are observed
both globally and locally.
The dynamics of the inner disk (i.e. within cylindrical radius $R \lesssim 5~\kpc$) is
likely dominated by a massive bar \citep[e.g.,][]{Blitz, Wegg} that induces resonant
substructure even out in the solar neighborhood \citep{todo}.
At intermediate radii in the disk ($5 \lesssim R \lesssim 12~\kpc$), there are
large-scale asymmetries in the vertical number counts \citep{Widrow:2012, Williams:2013}
and bulk motions \citep{Carlin??} of stars.
In this region, there is also substantial structure in the planar (radial and azimuthal)
kinematics of stars \citep{Katz:XX}, as clearly demonstrated with the transformative
astrometric data from data release 2 (\dr{2}) of the \gaia\ Mission
\citep{Gaia-dr2-papers}.
The outer disk ($R \gtrsim 12~\kpc$) shows even more significant substructure in the
form of phase-coherent feathers, stellar streams, and other features in stellar density
such as the Monoceros Ring, Anticenter Stream, Triangulum--Andromeda ``cloud,'' and
others \citep{Newberg, Slater, Price-Whelan, Xu, Li, Sheffield, Laporte}.
Many of these features are thought to be related to perturbations from satellite
galaxies, as simulations of satellite galaxy encounters with stellar disks produce
qualitatively similar signatures \citep[e.g.,][]{Old-paper, others, Laporte:2019}.

More locally, i.e. within small regions of the disk such as the solar neighborhood, the recently-discovered \gaia\ ``phase-space spiral'' is a striking kinematic feature that demonstrates the power of studying the Galaxy in phase-space (combined position and velocity), newly enabled with exquisite density resolution thanks to astrometric data from \gaia.
The spiral was initially discovered in the vertical kinematics ($z$--$v_z$) of stars using the number density and mean azimuthal velocity of stars with well-measured parallaxes from \gaia\ Data Release 2 (\dr{2}; \citealt{Antoja:2018}).
This feature --- and, more generally, any weak spiral in phase-space --- is understood to come from the partial phase mixing of a weakly-perturbed distribution function in a gravitational field with a gradient in the orbital frequencies \citep[e.g.,][]{Tremaine:1999, Binney:2018}.

In the case of the Milky Way, it is thought that an interaction with a satellite galaxy
(such as the Sagittarius dwarf galaxy) could be responsible for the perturbation(s) that
caused the vertical phase spirals \citep[e.g.,][]{Antoja:2018, Laporte:2019,
Darling:2019}.
Within this context, the spiral is a promising tool for .. studying the orbital history of Sag and mapping the mass of the disk.
Development of methods to use the spiral to this end.
Perturbation theory approach: response of distribution function in potential \citep{Banik}
?? approach: Widmark
?? approach: Frankel
In all cases, a fundamental parameter is time since phase mixing starts (in the simplest case, this is the time of impact)
Time can be connected to orbit of Sag, route to studying satellite merger history.

Outline: Another avenue toward measuring a time is to leverage stellar astrophysics. Ages notoriously hard to measure. Can use asteroseismology, but need high-quality time series. For spiral, signal is few percent perturbation, so need large sample. Most vetted asteroseismic samples are small by these standards (APOKASC? others). Large samples from TESS, but giants (Hon 2021), still selection effects because of time-series photometry requirement.

Outline: Alternate approach is to use main sequence lifetime. That's the basis of this work.


% The challenge of quantitatively matching the local phase-space spiral properties with simulations has motivated mapping the vertical spiral and its properties throughout larger observable portions of the Milky Way disk \citep[e.g.,][]{LAMOSTpaper, Hunt:2022, AntojaDR3}.


\section{Data} \label{sec:data}

We primarily use data from the \gaia\ Mission \citep{Gaia-Collaboration:2016}, Data
Release 3 (DR3; \citealt{}).
\gaia\ DR3 contains astrometric, spectroscopic, and photometric measurements along with
derived stellar parameters for hundreds of millions of stars throughout the Milky Way
(and beyond).
In this work, we make use of the astrometry \citep{todo}, ``GSP-Phot'' stellar
parameters \citep{todo}, and radial velocity measurements \citep{todo}.

Our initial sample is selected from \gaia\ DR3 with the following criteria:
\begin{itemize}
    \item High-quality astrometric measurements, $\textsf{RUWE} < 1.4$,
    \item Measured radial velocity, $-1000~\kms < v_r < 1000~\kms$,
    \item Near the sun; parallax $\varpi > 0.25~\mas$,
    \item High astrometric fidelity, $\mathtt{fidelity\_v2} > 0.5$ (as described in
    \citealt{Rybizki:2022}),
    \item Measured effective temperature $\Teff$ and $\logg$ from the GSP-Phot pipeline.
\end{itemize}
We then construct our parent sample by selecting out of this sample a cube with side
length $5.6~\kpc$ centered on the sun.
We do this by transforming the initial sample sky position and distance (computed as
$d=\varpi^{-1}$) to Galactocentric coordinates adopting sun–Galactic center distance of
$R_0 = 8.275~\kpc$ \citep{Gravity-Collaboration:2021}, a null Solar position above the
midplane (to be measured later), a radial velocity of Sgr A* of $-8.4~\kms$ (this value
is computed using the reported $z_0 = -2.6~\kpc$ and adding their fiducial value of
$11~\kms$; \citealt{Gravity-Collaboration:2021}), and a proper motion of Sgr A* of
$(\mu_{\alpha,*}, \mu_\delta) = (-3.16, -5.59)~\masyr$ \citep{Reid:2020}. These values
correspond to a total Solar velocity with respect to the Galactic center of
$\boldsymbol{v}_\odot = (8.4, 251.8, 8.4)~\kms$.

\texttt{rv\_template\_teff} < 14,500~K


Some extra words about effective temperatures and stellar parameters, new stuff in DR3.

Figure 1: Show the dependence of z-vz spiral with effective temperature selections.
- Comments about how spiral changes with effective temperature, proxy for age selection
- Comments about how naive visualizations that cut by guiding radius include selection effects because low-mass (preferentially older) stars only seen locally, high-mass (preferentially younger) stars seen further.
- Make sure that in my selections, lowest temp bin is still complete at 4 kpc

\todo{If we decide to include it, also words about masses (and implied ages) from asteroseismology.}

Stellar models to go from effective temperature / mass to lifetime.

% \subsection{Ages}
% In order to identify the timing of the creation of the phase spiral, and its potential connection to the Sagittarius dwarf galaxy, we need a large sample of stars with well constrained ages. While asteroseismic ages are the most precise and accurate option, previous space-based time series photometry missions like \kepler \citep{Borucki2010} and \ktwo\ \citep{someone} only targeted small numbers of stars in particular fields of the sky \citep{Pinsonneault2014, Pinsonneault2018, Pinsonneault2022, Stello2017, Zinn2020, Zinn2021}. The recent launch of the \tess mission \citep{Ricker:2014}, however, has made asteroseismology possible for large samples of stars across the sky. Initial investigations have demonstrated the potential of \tess\ for asteroseismology with single stars and restricted fields \citep{Huber2019, Mackereth2021}, but the all-sky search for giants is only just beginning.

% \citet{Hon2021} represents the first attempt to identify and characterize large numbers of oscillating red giants across the prime mission (2 years, 26 sectors) in \tess. This analysis builds upon the machine learning efforts used in \kepler \citep{Hon201?} training a neural net to identify potential oscillations signals in the pictures of the Fourier transforms of QLP \citep{someone?} light curves. In the process of this analysis, the pipeline makes an estimate of the frequency of maximum oscillation power \numax. In combination with data from \gaia\ \citep{Gaia-Collaboration:2018}, which allows an estimate of the radius and temperature of these relatively nearby giants, one can estimate a mass for each of the 180,000 stars in the sample. \citet{Hon2021} caution that these masses may have larger scatter and a higher rate of significant errors than the sorts of asteroseismic results that have previously been published for \kepler\ and \ktwo\. However, some of the initial analysis shown in \citet{Hon2021} suggests that the ensemble of masses is sufficiently accurate to identify galactic structures including a younger thin disk plane, phase space ridges, and the correlation between mass and velocity dispersion.

% Following this analysis, we use the mass proxies computed from the \citet{Hon2021} data to divide stars into coarse age bins. Specifically, we assume that all stars identified as less than 1.4 \msun\ are old ($>$ 4 Gyr), stars between 1.4 and 2.0 \msun\ are moderate age (between 1 and 4 Gyr), stars above 2.0 \msun\ are young ($<$ 1 Gyr), and stars above 3 \msun\ are very young ($<$500 Myr).

% Formally, estimating ages would require composition information as well as precise masses. Here we do not have metallicities for most of the stars, but we use the overlap sample with APOGEE Data Release 16 \citep{DR16} to argue that particularly for the more massive, younger stars of interest here, the stars are of order solar metallicity, and that given the uncertainties on our masses and our relatively course age bins, such an assumption does not bias our results at a significant level.

\section{Methods} \label{sec:methods}

TODO: Overview.

\subsection{Selecting stars with similar orbits}
\label{sec:methods-select-similar-orbit}

A challenge in interpreting the observed spatial variations of the spiral is that any spatially-selected region of the disk contains stars on many different types of orbits.
For example, in a given radially- and azimuthally-selected boxel, there exist a mixture of circular orbits with radii within the radial selection, and eccentric orbits with guiding radii outside of this region, but with apo- or peri-centric radii that bring them within this radial selection.
The implication of this fact is that stars in a given \emph{spatial} selection may have very different orbital histories, and therefore may have experienced different perturbations, and have had more or less time to phase mix relative to others in order to end up in the spatial region today.
In other words, any spatially-selected region of the disk contains many different, superimposed phase spirals that could correspond to different initial perturbations and subsequent orbital histories.

A key ... noted in \citep{LaporteXX}

Outline: Understanding of dynamical separation of spatial samples (Hunt, Gandhi). Characterize kinematic dependence of spiral (Hunt 2022), demonstration of different spiral morphologies. Method for constraining disk properties (Widmark). Perturbation timescale, amplitude (Darragh-Ford).

\subsection{Empirical models of the vertical orbit structure (without assuming a gravitational potential)}
\label{sec:zvz-model}

Crap about fitting the DF with my $r_z$ and $m=2$, $m=4$ distortions.

Assumptions

\begin{description}
    \item[Star formation history] Constant star formation rate
    \item[Selection effects] Ignored
    \item[Common orbital history] Stars within guiding radius selection traveled together
    \item[Initial mass function] Whatever - is this needed?
\end{description}


\subsection{Quantifying the Amplitude of Non-phase-mixed Features}
\label{sec:residuals}

Matched filter, A/B split idea.

\section{Results: Maps of amplitude of non-phase-mixed structure}
\label{sec:results-maps}

\section{Results: Spiral Onset Time as a Function of ...}
\label{sec:results-spiral-time}


\section{Discussion} \label{sec:discussion}

\subsection{Is the Sagittarius impact scenario too idealized?}
\label{sec:sag-impact}

Many tailored galaxy simulations with Sagittarius-motivated satellite perturbers have successfully produced phase-space spiral features in the simulated disks that qualitatively match the amplitude and morphology of the observed spiral \citep{Khanna:2019, Laporte:2019, Bland-Hawthorn:2021, Hunt:2021, Gandhi:2022}.
Quantitatively matching the observed amplitude of asymmetry in the Milky Way with $N$-body simulations of a simulated Sagittarius-like satellite and a Galactic disk that starts from an equilibrium state has not been successful \citep{Bennett:2022}, or requires a Sagittarius mass that is inconsistent with recent constraints on the present-day mass of Sagittarius \citep{Vasiliev:XX}.
It is therefore possible that multiple perturbations are required to produce the observed phase-space spirals \citep[e.g.,][]{Garcia-Conde:2022}, or that the response of the inner dark matter halo to the orbit of Sagittarius is an significant component of the perturbation \citep{Grand:2022}.
Still interesting prospect: can we see when one spiral goes away and another is left?

Or, maybe not Sag and instead multiple mergers / noise \citep{Tremaine:2022}.
In that scenario, spirals should always exist.


\section{Conclusions} \label{sec:conclusions}


\begin{acknowledgements}

It is a pleasure to thank ...

% Funding for the Sloan Digital Sky Survey IV has been provided by the Alfred P.
% Sloan Foundation, the U.S. Department of Energy Office of Science, and the
% Participating Institutions. SDSS-IV acknowledges support and resources from the
% Center for High-Performance Computing at the University of Utah. The SDSS web
% site is www.sdss.org.

% SDSS-IV is managed by the Astrophysical Research Consortium for the
% Participating Institutions of the SDSS Collaboration including the Brazilian
% Participation Group, the Carnegie Institution for Science, Carnegie Mellon
% University, the Chilean Participation Group, the French Participation Group,
% Harvard-Smithsonian Center for Astrophysics, Instituto de Astrof\'isica de
% Canarias, The Johns Hopkins University, Kavli Institute for the Physics and
% Mathematics of the Universe (IPMU) / University of Tokyo, Lawrence Berkeley
% National Laboratory, Leibniz Institut f\"ur Astrophysik Potsdam (AIP),
% Max-Planck-Institut f\"ur Astronomie (MPIA Heidelberg), Max-Planck-Institut
% f\"ur Astrophysik (MPA Garching), Max-Planck-Institut f\"ur Extraterrestrische
% Physik (MPE), National Astronomical Observatories of China, New Mexico State
% University, New York University, University of Notre Dame, Observat\'ario
% Nacional / MCTI, The Ohio State University, Pennsylvania State University,
% Shanghai Astronomical Observatory, United Kingdom Participation Group,
% Universidad Nacional Aut\'onoma de M\'exico, University of Arizona, University
% of Colorado Boulder, University of Oxford, University of Portsmouth, University
% of Utah, University of Virginia, University of Washington, University of
% Wisconsin, Vanderbilt University, and Yale University.

This work has made use of data from the European Space Agency (ESA) mission
{\it Gaia} (\url{https://www.cosmos.esa.int/gaia}), processed by the {\it Gaia}
Data Processing and Analysis Consortium (DPAC,
\url{https://www.cosmos.esa.int/web/gaia/dpac/consortium}). Funding for the DPAC
has been provided by national institutions, in particular the institutions
participating in the {\it Gaia} Multilateral Agreement.

\end{acknowledgements}

\software{
    Astropy \citep{astropy:2013, astropy:2018, astropy:2022},
    gala \citep{gala},
    IPython \citep{ipython},
    numpy \citep{numpy},
    % pymc3 \citep{Salvatier2016},
    % schwimmbad \citep{schwimmbad:2017},
    scipy \citep{scipy}.
}

\bibliographystyle{aasjournal}
\bibliography{snail-time}

\end{document}
