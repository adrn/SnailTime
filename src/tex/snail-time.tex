% Notes:
% -

% Relevant papers (for bibliograph):
% https://arxiv.org/pdf/1903.01493.pdf
% https://arxiv.org/pdf/2009.02434.pdf
% https://arxiv.org/pdf/1809.02658.pdf
% https://arxiv.org/pdf/1902.10113.pdf
% https://arxiv.org/pdf/2011.02490.pdf
% https://arxiv.org/pdf/1808.00451.pdf

% \begin{figure}[!t]
% \begin{center}
% % \includegraphics[width=0.9\textwidth]{visitstats.pdf}
% {\color{red} Figure placeholder}
% \end{center}
% \caption{%
% TODO
% \label{fig:chiplots}
% }
% \end{figure}

\PassOptionsToPackage{usenames,dvipsnames}{xcolor}
\documentclass[modern]{aastex631}
% \documentclass[twocolumn]{aastex631}
\usepackage{showyourwork}

% Load common packages
\usepackage{microtype}  % ALWAYS!
\usepackage{amsmath}
\usepackage{amsfonts}
\usepackage{amssymb}
\usepackage{booktabs}
\usepackage{graphicx}
% \usepackage{color}

\usepackage{enumitem}
\setlist[description]{style=unboxed}

% Some style hacks:
\renewcommand{\twocolumngrid}{\onecolumngrid}
\setlength{\parindent}{1.1\baselineskip}
\addtolength{\topmargin}{-0.2in}
\addtolength{\textheight}{0.4in}
\sloppy\sloppypar\raggedbottom\frenchspacing

\graphicspath{{figures/}}
% \definecolor{cbblue}{HTML}{3182bd}
% \usepackage{hyperref}
% \definecolor{linkcolor}{rgb}{0.02,0.35,0.55}
% \definecolor{citecolor}{rgb}{0.45,0.45,0.45}
% \hypersetup{colorlinks=true,linkcolor=linkcolor,citecolor=citecolor,
%             filecolor=linkcolor,urlcolor=linkcolor}
% \hypersetup{pageanchor=true}

\newcommand{\documentname}{\textsl{Article}}
\newcommand{\sectionname}{Section}
\renewcommand{\figurename}{Figure}
\newcommand{\equationname}{Equation}
\renewcommand{\tablename}{Table}

% Missions
\newcommand{\project}[1]{\textsl{#1}}

% Packages / projects / programming
\newcommand{\package}[1]{\textsl{#1}}
\newcommand{\acronym}[1]{{\small{#1}}}
\newcommand{\github}{\package{GitHub}}
\newcommand{\python}{\package{Python}}
\newcommand{\emcee}{\project{emcee}}

% Stats / probability
\newcommand{\given}{\,|\,}
\newcommand{\norm}{\mathcal{N}}
\newcommand{\pdf}{\textsl{pdf}}

% Maths
\newcommand{\dd}{\mathrm{d}}
\newcommand{\transpose}[1]{{#1}^{\mathsf{T}}}
\newcommand{\inverse}[1]{{#1}^{-1}}
\newcommand{\argmin}{\operatornamewithlimits{argmin}}
\newcommand{\mean}[1]{\left< #1 \right>}

% Non-scalar variables
\renewcommand{\vec}[1]{\ensuremath{\bs{#1}}}
\newcommand{\mat}[1]{\ensuremath{\mathbf{#1}}}

% Unit shortcuts
\newcommand{\msun}{\ensuremath{\mathrm{M}_\odot}}
\newcommand{\mjup}{\ensuremath{\mathrm{M}_{\mathrm{J}}}}
\newcommand{\kms}{\ensuremath{\mathrm{km}~\mathrm{s}^{-1}}}
\newcommand{\mps}{\ensuremath{\mathrm{m}~\mathrm{s}^{-1}}}
\newcommand{\pc}{\ensuremath{\mathrm{pc}}}
\newcommand{\kpc}{\ensuremath{\mathrm{kpc}}}
\newcommand{\kmskpc}{\ensuremath{\mathrm{km}~\mathrm{s}^{-1}~\mathrm{kpc}^{-1}}}
\newcommand{\dayd}{\ensuremath{\mathrm{d}}}
\newcommand{\yr}{\ensuremath{\mathrm{yr}}}
\newcommand{\AU}{\ensuremath{\mathrm{AU}}}
\newcommand{\Kel}{\ensuremath{\mathrm{K}}}

% Misc.
\newcommand{\bs}[1]{\boldsymbol{#1}}

% Astronomy
\newcommand{\DM}{{\rm DM}}
\newcommand{\feh}{\ensuremath{{[{\rm Fe}/{\rm H}]}}}
\newcommand{\mh}{\ensuremath{{[{\rm M}/{\rm H}]}}}
\newcommand{\df}{\acronym{DF}}
\newcommand{\logg}{\ensuremath{\log g}}
\newcommand{\Teff}{\ensuremath{T_{\textrm{eff}}}}
\newcommand{\vsini}{\ensuremath{v\,\sin i}}
\newcommand{\mtwomin}{\ensuremath{M_{2, {\rm min}}}}

% TO DO
\newcommand{\todo}[1]{{\color{red} TODO: #1}}

\newcommand{\gaia}{\textsl{Gaia}}
\newcommand{\dr}[1]{\acronym{DR}#1}
\newcommand{\apogee}{\acronym{APOGEE}}
\newcommand{\sdss}{\acronym{SDSS}}
\newcommand{\sdssiv}{\acronym{SDSS-IV}}
\newcommand{\thejoker}{\project{The~Joker}}

% Custom:
\newcommand{\kepler}{\project{Kepler}}
\newcommand{\ktwo}{\project{K2}}
\newcommand{\tess}{\acronym{TESS}}
\newcommand{\numax}{\ensuremath{\nu_{\rm max}}}

\shorttitle{}
\shortauthors{Price-Whelan et al.}

\begin{document}

\title{Timing Vertical Phase Spirals with Stellar Lifetimes}

\newcommand{\affcca}{
    Center for Computational Astrophysics, Flatiron Institute,
    162 Fifth Ave, New York, NY 10010, USA
}

\author[0000-0003-0872-7098]{Adrian~M.~Price-Whelan}
\affiliation{\affcca}
\email{aprice-whelan@flatironinstitute.org}
\correspondingauthor{Adrian M. Price-Whelan}

% TODO: orcid
\author{Jason A. S. Hunt}
\affiliation{\affcca}

% TODO: orcid, aff
% \author{Elise Darragh-Ford}
% \affiliation{\affstanford}
% \affiliation{\affcca}
\
% TODO: orcid, affs
% \author{David~W.~Hogg}
% \affiliation{\affcca}
% \affiliation{\affnyu}
% \affiliation{\affmpia}

% TODO: orcid, aff
% \author{Kathryn Johnston}
% \affiliation{\affcolumbia}

\author{+ more}


\begin{abstract}\noindent
% Context
The discovery of phase-space spiral features in the Milky Way disk --- made possible by exquisite stellar astrometry from the \gaia\ Mission --- has enabled new methods for constraining the mass distribution and perturbation history of the Galaxy.
These spiral features are thought to form as a result of phase mixing of orbits with different characteristic frequencies as a result of weak perturbations to the distribution function.
As a result, the amplitude, morphology, and global dependence (over the Galactic disk), of vertical ($z$--$v_z$) phase spirals is determined by the time, timescales, and strengths of perturbations to the Milky Way and the resulting response of the system.
Mapping and dynamically ``rewinding'' the observed phase spirals has therefore led to new insights about the orbit and interactions of the Sagittarius dwarf galaxy (a significant perturber of the Milky Way) and the subsequent evolution of the Galaxy.
% Aims
In this work, we aim to measure the onset time of vertical phase spirals in dynamically-separated populations of stars that are presently near the Sun by exploiting main-sequence stellar lifetimes rather than dynamical phase mixing.
% Methods
We use stellar temperatures derived from \gaia\ Data Release 3 (\dr{3}) and main sequence lifetimes predicted from the \placeholder{TODO} stellar models to split our stellar samples into subpopulations with consistent maximum stellar lifetimes.
In these subpopulations, we measure the amplitude of non-phase-mixed structure in the vertical kinematics using empirical matched filters tuned to the orbit structure of a given dynamical patch, without assuming a form for the gravitational potential of the Milky Way.
% Results
We find that the ...
% Conclusions
This work presents a new approach for quantifying the onset time of vertical phase-space spirals in the Milky Way leveraging stellar physics rather than dynamics.


\end{abstract}

% \keywords{}

\section*{~}\clearpage
\section{Introduction} \label{sec:intro}

% Milky Way context and mergers and cosomology crap. Want to measure dark matter/mass distribution, and formation history/evolution of galaxy.
The Milky Way is a unique laboratory for studying the dynamical processes that shape galaxies and the detailed structure of dark matter within a galaxy.
It is the one galaxy where contemporary stellar surveys measure kinematics and stellar parameters for hundreds of millions to billions of individual stars at all evolutionary stages, and where our perspective within the Galaxy grants us a three-dimensional view of its stars (and gas).
Because of its importance as our home and as a benchmark system for contextualizing studies of external galaxies, significant effort has gone into modeling the observed distribution and properties of its stellar populations \citep[for a review,][]{Bland-Hawthron:2016} and using these observations to build dynamical models of its structure and evolution \citep{TODO galactic dynamics summary}.
Even with this exquisite data, the fact that we observe the system at (effectively) a single dynamical snapshot means that many of these measurements and interpretations are built on assumptions that the Galaxy is (at least approximately) in steady state, or that any perturbations and time-dependent processes can be (statistically) averaged over.
In recent years, owing to precise and expansive data from stellar surveys like the \gaia\ Mission \citep{Gaia-overview}, it is now thought that many of these assumptions are significantly violated.

% However, explicitly modeling these departures from steady state provide promising new directions for precisely constraining the dark matter distribution around the Galaxy, measuring the global structural parameters of the stellar components of the Milky Way, and inferring the mass assembly history and dominant dynamical processes.
% A critical step toward making these precise measurements is to characterize the TODO ... of signatures of disequilibrium ... in data.

Recent observations and simulations of the Galaxy have demonstrated the existence or expectation of significant un-mixed kinematic substructure at all spatial scales in the Galaxy.
For example, on the largest scales, ... LMC \citep[e.g.,][]{Garavito-Camargo, blah}.
In the Milky Way disk --- the focus of this work --- there are observed signatures of disequilibrium at all radii.
In the inner disk (within cylindrical radius $R \lesssim XX~\kpc$), there are large-scale asymmetries in the vertical number counts \citep{Widrow??} and bulk motions \citep{Carlin??} of stars.
In this region, there is also substantial structure in the planar (radial and azimuthal) kinematics of stars \citep{Katz:XX}, as clearly demonstrated with the transformative astrometric data from data release 2 (\dr{2}) of the \gaia\ Mission \citep{Gaia-dr2-papers}.
The outer disk ($R \gtrsim X~\kpc$) shows even more significant substructure in the form of phase-coherent feathers, stellar streams, and other features in stellar density such as the Monoceros Ring, Anticenter Stream, Triangulum--Andromeda ``cloud,'' and others \citep{Newberg, Slater, Price-Whelan, Xu, Li, Sheffield, Laporte}.
All of these features are thought to be related to perturbations from satellite galaxies, as simulations of satellite galaxy encounters with stellar disks produce qualitatively similar effects \citep[e.g.,][]{Old-paper,others, Laporte:2019}.

On smaller scales, the recently-discovered \gaia\ ``phase-space spiral'' is a striking kinematic feature that demonstrates the power of studying the Galaxy in phase-space (combined position and velocity), newly enabled with exquisite density resolution thanks to astrometric data from \gaia.
The spiral was initially discovered in the vertical kinematics ($z$--$v_z$) of stars using the number density and mean azimuthal velocity of stars with well-measured parallaxes from \gaia\ Data Release 2 (\dr{2}; \citealt{Antoja:2018}).
This feature --- and, more generally, any weak spiral in phase-space --- is understood to come from the partial phase mixing of a weakly-perturbed distribution function in a gravitational field with a gradient in the orbital frequencies \citep[e.g.,][]{Binney:2018}.

In the case of the Milky Way, it is thought that an interaction with a satellite galaxy (such as the Sagittarius dwarf galaxy) could be the responsible for the perturbation(s) that caused the spiral \citep[e.g.,][]{Antoja:2018, Laporte:2019, Darling:2019}.
Many tailored galaxy simulations with Sagittarius-motivated satellite perturbers have successfully produced phase-space spiral features in the simulated disks that qualitatively match the amplitude and morphology of the observed spiral \citep{Khanna:2019, Laporte:2019, Bland-Hawthorn:2021, Hunt:2021, Gandhi:2022}.
Quantitatively matching the observed amplitude of asymmetry in the Milky Way with $N$-body simulations of a simulated Sagittarius-like satellite and a Galactic disk that starts from an equilibrium state has not been successful \citep{Bennett:2022}, or requires a Sagittarius mass that is inconsistent with recent constraints on the present-day mass of Sagittarius \citep{Vasiliev:XX}.
It is therefore possible that multiple perturbations are required to produce the observed phase-space spirals \citep[e.g.,][]{Garcia-Conde:2022}, or that the response of the inner dark matter halo to the orbit of Sagittarius is an significant component of the perturbation \citep{Grand:2022}.

% Outline: Connects to bigger-picture perturbations to the disk. Galactoseismology as a path to studying merger history, dark matter physics, and evolution of galaxy. Asymmetries in number counts (Widrow), kinematics (Carlin). Outer disk features (Newberg, APW, Xu). All hint toward significant perturbations to disk (Laporte, Hunt sims).
Another important signature of perturbations to the Milky Way disk are observed, large-scale asymmetries in stellar number counts and kinematics of stars.

A goal of this effort is to gain a better understanding of the connection between these localized features and the global response of the disk, which ....



The challenge of quantitatively matching the local phase-space spiral properties with simulations has motivated mapping the vertical spiral and its properties throughout larger observable portions of the Milky Way disk \citep[e.g.,][]{XX}. % TODO: LAMOST paper

Outline: Understanding of dynamical separation of spatial samples (Hunt, Gandhi). Characterize kinematic dependence of spiral (Hunt 2022), demonstration of different spiral morphologies. Method for constraining disk properties (Widmark). Perturbation timescale, amplitude (Darragh-Ford).

Outline: Another avenue toward measuring a time of event is to leverage stellar astrophysics. Ages notoriously hard to measure. Can use asteroseismology, but need high-quality time series. For spiral, signal is few percent perturbation, so need large sample. Most vetted asteroseismic samples are small by these standards (APOKASC? others). Large samples from TESS, but giants (Hon 2021), still selection effects because of time-series photometry requirement.

Outline: Alternate approach is to use main sequence lifetime. That's the basis of this work.


\section{Data} \label{sec:data}

Primarily use data from Gaia DR3. Usual blurb and citations. Some extra words about effective temperatures and stellar parameters, new stuff in DR3.

Figure 1: Show the dependence of z-vz spiral with effective temperature.

\todo{If we decide to include it, also words about masses (and implied ages) from asteroseismology.}

Stellar models to go from effective temperature / mass to lifetime.

% \subsection{Ages}
% In order to identify the timing of the creation of the phase spiral, and its potential connection to the Sagittarius dwarf galaxy, we need a large sample of stars with well constrained ages. While asteroseismic ages are the most precise and accurate option, previous space-based time series photometry missions like \kepler \citep{Borucki2010} and \ktwo\ \citep{someone} only targeted small numbers of stars in particular fields of the sky \citep{Pinsonneault2014, Pinsonneault2018, Pinsonneault2022, Stello2017, Zinn2020, Zinn2021}. The recent launch of the \tess mission \citep{Ricker:2014}, however, has made asteroseismology possible for large samples of stars across the sky. Initial investigations have demonstrated the potential of \tess\ for asteroseismology with single stars and restricted fields \citep{Huber2019, Mackereth2021}, but the all-sky search for giants is only just beginning.

% \citet{Hon2021} represents the first attempt to identify and characterize large numbers of oscillating red giants across the prime mission (2 years, 26 sectors) in \tess. This analysis builds upon the machine learning efforts used in \kepler \citep{Hon201?} training a neural net to identify potential oscillations signals in the pictures of the Fourier transforms of QLP \citep{someone?} light curves. In the process of this analysis, the pipeline makes an estimate of the frequency of maximum oscillation power \numax. In combination with data from \gaia\ \citep{Gaia-Collaboration:2018}, which allows an estimate of the radius and temperature of these relatively nearby giants, one can estimate a mass for each of the 180,000 stars in the sample. \citet{Hon2021} caution that these masses may have larger scatter and a higher rate of significant errors than the sorts of asteroseismic results that have previously been published for \kepler\ and \ktwo\. However, some of the initial analysis shown in \citet{Hon2021} suggests that the ensemble of masses is sufficiently accurate to identify galactic structures including a younger thin disk plane, phase space ridges, and the correlation between mass and velocity dispersion.

% Following this analysis, we use the mass proxies computed from the \citet{Hon2021} data to divide stars into coarse age bins. Specifically, we assume that all stars identified as less than 1.4 \msun\ are old ($>$ 4 Gyr), stars between 1.4 and 2.0 \msun\ are moderate age (between 1 and 4 Gyr), stars above 2.0 \msun\ are young ($<$ 1 Gyr), and stars above 3 \msun\ are very young ($<$500 Myr).

% Formally, estimating ages would require composition information as well as precise masses. Here we do not have metallicities for most of the stars, but we use the overlap sample with APOGEE Data Release 16 \citep{DR16} to argue that particularly for the more massive, younger stars of interest here, the stars are of order solar metallicity, and that given the uncertainties on our masses and our relatively course age bins, such an assumption does not bias our results at a significant level.

\section{Methods} \label{sec:methods}

Overview.

\subsection{Empirical Models of the Phase-mixed $z$--$v_z$ Orbit Structure}
\label{sec:zvz-structure}

Crap about fitting the DF with my $r_z$ and $m=2$, $m=4$ distortions.


\subsection{Quantifying the Amplitude of Non-phase-mixed Features}
\label{sec:residuals}

Matched filter, A/B split idea.


\section{Results: Spiral Onset Time as a Function of ...}
\label{sec:results}


\section{Discussion} \label{sec:discussion}

\subsection{Limitations}
\label{sec:limitations}


\section{Conclusions} \label{sec:conclusions}


\begin{acknowledgements}

It is a pleasure to thank ...

% Funding for the Sloan Digital Sky Survey IV has been provided by the Alfred P.
% Sloan Foundation, the U.S. Department of Energy Office of Science, and the
% Participating Institutions. SDSS-IV acknowledges support and resources from the
% Center for High-Performance Computing at the University of Utah. The SDSS web
% site is www.sdss.org.

% SDSS-IV is managed by the Astrophysical Research Consortium for the
% Participating Institutions of the SDSS Collaboration including the Brazilian
% Participation Group, the Carnegie Institution for Science, Carnegie Mellon
% University, the Chilean Participation Group, the French Participation Group,
% Harvard-Smithsonian Center for Astrophysics, Instituto de Astrof\'isica de
% Canarias, The Johns Hopkins University, Kavli Institute for the Physics and
% Mathematics of the Universe (IPMU) / University of Tokyo, Lawrence Berkeley
% National Laboratory, Leibniz Institut f\"ur Astrophysik Potsdam (AIP),
% Max-Planck-Institut f\"ur Astronomie (MPIA Heidelberg), Max-Planck-Institut
% f\"ur Astrophysik (MPA Garching), Max-Planck-Institut f\"ur Extraterrestrische
% Physik (MPE), National Astronomical Observatories of China, New Mexico State
% University, New York University, University of Notre Dame, Observat\'ario
% Nacional / MCTI, The Ohio State University, Pennsylvania State University,
% Shanghai Astronomical Observatory, United Kingdom Participation Group,
% Universidad Nacional Aut\'onoma de M\'exico, University of Arizona, University
% of Colorado Boulder, University of Oxford, University of Portsmouth, University
% of Utah, University of Virginia, University of Washington, University of
% Wisconsin, Vanderbilt University, and Yale University.

This work has made use of data from the European Space Agency (ESA) mission
{\it Gaia} (\url{https://www.cosmos.esa.int/gaia}), processed by the {\it Gaia}
Data Processing and Analysis Consortium (DPAC,
\url{https://www.cosmos.esa.int/web/gaia/dpac/consortium}). Funding for the DPAC
has been provided by national institutions, in particular the institutions
participating in the {\it Gaia} Multilateral Agreement.

\end{acknowledgements}

\software{
    Astropy \citep{astropy:2013, astropy:2018, astropy:2022},
    gala \citep{gala},
    IPython \citep{ipython},
    numpy \citep{numpy},
    % pymc3 \citep{Salvatier2016},
    % schwimmbad \citep{schwimmbad:2017},
    scipy \citep{scipy}.
}

\bibliographystyle{aasjournal}
\bibliography{snail-time}

\end{document}
