% Notes:
% -

% Relevant papers:
% https://arxiv.org/pdf/1903.01493.pdf
% https://arxiv.org/pdf/2009.02434.pdf
% https://arxiv.org/pdf/1809.02658.pdf
% https://arxiv.org/pdf/1902.10113.pdf
% https://arxiv.org/pdf/2011.02490.pdf
% https://arxiv.org/pdf/1808.00451.pdf

% \begin{figure}[!t]
% \begin{center}
% % \includegraphics[width=0.9\textwidth]{visitstats.pdf}
% {\color{red} Figure placeholder}
% \end{center}
% \caption{%
% TODO
% \label{fig:chiplots}
% }
% \end{figure}

\PassOptionsToPackage{usenames,dvipsnames}{xcolor}
\documentclass[modern]{aastex631}
% \documentclass[twocolumn]{aastex63}

% Load common packages
\usepackage{microtype}  % ALWAYS!
\usepackage{amsmath}
\usepackage{amsfonts}
\usepackage{amssymb}
\usepackage{booktabs}
\usepackage{graphicx}
% \usepackage{color}

\usepackage{enumitem}
\setlist[description]{style=unboxed}

% Hogg's issues
\renewcommand{\twocolumngrid}{\onecolumngrid} % guess what this does HAHAHA!
\setlength{\parindent}{1.1\baselineskip}
\addtolength{\topmargin}{-0.2in}
\addtolength{\textheight}{0.4in}
\sloppy\sloppypar\raggedbottom\frenchspacing

% For referee:
\newcommand{\changes}[1]{{\color{violet}#1}}


\graphicspath{{figures/}}
% \definecolor{cbblue}{HTML}{3182bd}
% \usepackage{hyperref}
% \definecolor{linkcolor}{rgb}{0.02,0.35,0.55}
% \definecolor{citecolor}{rgb}{0.45,0.45,0.45}
% \hypersetup{colorlinks=true,linkcolor=linkcolor,citecolor=citecolor,
%             filecolor=linkcolor,urlcolor=linkcolor}
% \hypersetup{pageanchor=true}

\newcommand{\documentname}{\textsl{Article}}
\newcommand{\sectionname}{Section}
\renewcommand{\figurename}{Figure}
\newcommand{\equationname}{Equation}
\renewcommand{\tablename}{Table}

% Missions
\newcommand{\project}[1]{\textsl{#1}}

% Packages / projects / programming
\newcommand{\package}[1]{\textsl{#1}}
\newcommand{\acronym}[1]{{\small{#1}}}
\newcommand{\github}{\package{GitHub}}
\newcommand{\python}{\package{Python}}
\newcommand{\emcee}{\project{emcee}}

% Stats / probability
\newcommand{\given}{\,|\,}
\newcommand{\norm}{\mathcal{N}}
\newcommand{\pdf}{\textsl{pdf}}

% Maths
\newcommand{\dd}{\mathrm{d}}
\newcommand{\transpose}[1]{{#1}^{\mathsf{T}}}
\newcommand{\inverse}[1]{{#1}^{-1}}
\newcommand{\argmin}{\operatornamewithlimits{argmin}}
\newcommand{\mean}[1]{\left< #1 \right>}

% Non-scalar variables
\renewcommand{\vec}[1]{\ensuremath{\bs{#1}}}
\newcommand{\mat}[1]{\ensuremath{\mathbf{#1}}}

% Unit shortcuts
\newcommand{\msun}{\ensuremath{\mathrm{M}_\odot}}
\newcommand{\mjup}{\ensuremath{\mathrm{M}_{\mathrm{J}}}}
\newcommand{\kms}{\ensuremath{\mathrm{km}~\mathrm{s}^{-1}}}
\newcommand{\mps}{\ensuremath{\mathrm{m}~\mathrm{s}^{-1}}}
\newcommand{\pc}{\ensuremath{\mathrm{pc}}}
\newcommand{\kpc}{\ensuremath{\mathrm{kpc}}}
\newcommand{\kmskpc}{\ensuremath{\mathrm{km}~\mathrm{s}^{-1}~\mathrm{kpc}^{-1}}}
\newcommand{\dayd}{\ensuremath{\mathrm{d}}}
\newcommand{\yr}{\ensuremath{\mathrm{yr}}}
\newcommand{\AU}{\ensuremath{\mathrm{AU}}}
\newcommand{\Kel}{\ensuremath{\mathrm{K}}}

% Misc.
\newcommand{\bs}[1]{\boldsymbol{#1}}

% Astronomy
\newcommand{\DM}{{\rm DM}}
\newcommand{\feh}{\ensuremath{{[{\rm Fe}/{\rm H}]}}}
\newcommand{\mh}{\ensuremath{{[{\rm M}/{\rm H}]}}}
\newcommand{\df}{\acronym{DF}}
\newcommand{\logg}{\ensuremath{\log g}}
\newcommand{\Teff}{\ensuremath{T_{\textrm{eff}}}}
\newcommand{\vsini}{\ensuremath{v\,\sin i}}
\newcommand{\mtwomin}{\ensuremath{M_{2, {\rm min}}}}

% TO DO
\newcommand{\todo}[1]{{\color{red} TODO: #1}}

\newcommand{\gaia}{\textsl{Gaia}}
\newcommand{\dr}[1]{\acronym{DR}#1}
\newcommand{\apogee}{\acronym{APOGEE}}
\newcommand{\sdss}{\acronym{SDSS}}
\newcommand{\sdssiv}{\acronym{SDSS-IV}}
\newcommand{\thejoker}{\project{The~Joker}}

\shorttitle{}
\shortauthors{Price-Whelan et al.}

\begin{document}

\title{A First Look at Timing the Gaia Phase Spiral with Asteroseismology}

\author{People}

% \author[0000-0003-0872-7098]{Adrian~M.~Price-Whelan}
% \affiliation{Center for Computational Astrophysics, Flatiron Institute,
%              Simons Foundation, 162 Fifth Avenue, New York, NY 10010, USA}
% \email{aprice-whelan@flatironinstitute.org}
% \correspondingauthor{Adrian M. Price-Whelan}


\begin{abstract}\noindent
TODO
\end{abstract}

% \keywords{}

\section*{~}\clearpage
\section{Introduction} \label{sec:intro}

Stuff.


\section{Data} \label{sec:data}

Things.
\subsection{Ages}
In order to identify the timing of the creation of the phase spiral, and its potential connection to the Sagittarius dwarf galaxy, we need a large sample of stars with well constrained ages. While asteroseismic ages are the most precise and accurate option, previous space-based time series photometry missions like \kepler\ \citep{Borucki2010} and \ktwo\ \citep{someone} only targeted small numbers of stars in particular fields of the sky \citep{Pinsonneault2014, Pinsonneault2018, Pinsonneault2022, Stello2017, Zinn2020, Zinn2021}. The recent launch of the \tess mission \citep{Ricker:2014}, however, has made asteroseismology possible for large samples of stars across the sky. Initial investigations have demonstrated the potential of \tess\ for asteroseismology with single stars and restricted fields \citep{Huber2019, Mackereth2021}, but the all-sky search for giants is only just beginning.

\citet{Hon2021} represents the first attempt to identify and characterize large numbers of oscillating red giants across the prime mission (2 years, 26 sectors) in \tess. This analysis builds upon the machine learning efforts used in \kepler \citep{Hon201?} training a neural net to identify potential oscillations signals in the pictures of the Fourier transforms of QLP \citep{someone?} light curves. In the process of this analysis, the pipeline makes an estimate of the frequency of maximum oscillation power \numax. In combination with data from \gaia\ \citep{Gaia-Collaboration:2018}, which allows an estimate of the radius and temperature of these relatively nearby giants, one can estimate a mass for each of the 180,000 stars in the sample. \citet{Hon2021} caution that these masses may have larger scatter and a higher rate of significant errors than the sorts of asteroseismic results that have previously been published for \kepler\ and \ktwo\. However, some of the initial analysis shown in \citet{Hon2021} suggests that the ensemble of masses is sufficiently accurate to identify galactic structures including a younger thin disk plane, phase space ridges, and the correlation between mass and velocity dispersion.

Following this analysis, we use the mass proxies computed from the \citet{Hon2021} data to divide stars into coarse age bins. Specifically, we assume that all stars identified as less than 1.4 \msun\ are old ($>$ 4 Gyr), stars between 1.4 and 2.0 \msun\ are moderate age (between 1 and 4 Gyr), stars above 2.0 \msun\ are young ($<$ 1 Gyr), and stars above 3 \msun\ are very young ($<$500 Myr).

Formally, estimating ages would require composition information as well as precise masses. Here we do not have metallicities for most of the stars, but we use the overlap sample with APOGEE Data Release 16 \citep{DR16} to argue that particularly for the more massive, younger stars of interest here, the stars are of order solar metallicity, and that given the uncertainties on our masses and our relatively course age bins, such an assumption does not bias our results at a significant level.

\section{Methods} \label{sec:methods}

\subsection{Defining an Asymmetry Parameter to Quantify Spirality}

People have used an asymmetry parameter defined as
\begin{equation}
    A_X = \frac{N(X>0) - N(X\leq0)}{N(X>0) + N(X\leq0)}
\end{equation}
to quantify the strength of perturbation. But this is not a very sensitive way
of assessing whether there is a \emph{spiral} pattern versus more global
asymmetries.

\section{Results} \label{sec:results}

\section{Discussion} \label{sec:discussion}

\section{Conclusions} \label{sec:conclusions}


\acknowledgements

It is a pleasure to thank ...

% Funding for the Sloan Digital Sky Survey IV has been provided by the Alfred P.
% Sloan Foundation, the U.S. Department of Energy Office of Science, and the
% Participating Institutions. SDSS-IV acknowledges support and resources from the
% Center for High-Performance Computing at the University of Utah. The SDSS web
% site is www.sdss.org.

% SDSS-IV is managed by the Astrophysical Research Consortium for the
% Participating Institutions of the SDSS Collaboration including the Brazilian
% Participation Group, the Carnegie Institution for Science, Carnegie Mellon
% University, the Chilean Participation Group, the French Participation Group,
% Harvard-Smithsonian Center for Astrophysics, Instituto de Astrof\'isica de
% Canarias, The Johns Hopkins University, Kavli Institute for the Physics and
% Mathematics of the Universe (IPMU) / University of Tokyo, Lawrence Berkeley
% National Laboratory, Leibniz Institut f\"ur Astrophysik Potsdam (AIP),
% Max-Planck-Institut f\"ur Astronomie (MPIA Heidelberg), Max-Planck-Institut
% f\"ur Astrophysik (MPA Garching), Max-Planck-Institut f\"ur Extraterrestrische
% Physik (MPE), National Astronomical Observatories of China, New Mexico State
% University, New York University, University of Notre Dame, Observat\'ario
% Nacional / MCTI, The Ohio State University, Pennsylvania State University,
% Shanghai Astronomical Observatory, United Kingdom Participation Group,
% Universidad Nacional Aut\'onoma de M\'exico, University of Arizona, University
% of Colorado Boulder, University of Oxford, University of Portsmouth, University
% of Utah, University of Virginia, University of Washington, University of
% Wisconsin, Vanderbilt University, and Yale University.

This work has made use of data from the European Space Agency (ESA) mission
{\it Gaia} (\url{https://www.cosmos.esa.int/gaia}), processed by the {\it Gaia}
Data Processing and Analysis Consortium (DPAC,
\url{https://www.cosmos.esa.int/web/gaia/dpac/consortium}). Funding for the DPAC
has been provided by national institutions, in particular the institutions
participating in the {\it Gaia} Multilateral Agreement.

\software{
    Astropy \citep{astropy, astropy:2018},
    gala \citep{gala},
    IPython \citep{ipython},
    numpy \citep{numpy},
    % pymc3 \citep{Salvatier2016},
    % schwimmbad \citep{schwimmbad:2017},
    scipy \citep{scipy}.
}

\bibliographystyle{aasjournal}
\bibliography{phase-spiral-astero}

\end{document}
